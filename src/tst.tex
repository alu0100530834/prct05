\documentclass{beamer}
\usepackage[spanish]{babel}
\usepackage[utf8]{inputenc}
\usepackage{graphicx}

\newtheorem{definicion}{Definición}
\newtheorem{ejemplo}{Ejemplo}

%%%%%%%%%%%%%%%%%%%%%%%%%%%%%%%%%%%%%%%%%%%%%%%%%%%%%%%%%%%%%%%%%%%%%%%%%%%%%%%
\title[Presentación con Beamer]{Introduccion al \LaTeX{} }
\author[L. Diaz]{Lorena Díaz Morales}
\date[15-03-2013]{15 de marzo de 2013}
%%%%%%%%%%%%%%%%%%%%%%%%%%%%%%%%%%%%%%%%%%%%%%%%%%%%%%%%%%%%%%%%%%%%%%%%%%%%%%%

\usetheme{Madrid}
%\usetheme{Antibes}
%\usetheme{boxes}
%\usetheme{tree}
%\usetheme{classic}

%%%%%%%%%%%%%%%%%%%%%%%%%%%%%%%%%%%%%%%%%%%%%%%%%%%%%%%%%%%%%%%%%%%%%%%%%%%%%%%
\begin{document}
  
%++++++++++++++++++++++++++++++++++++++++++++++++++++++++++++++++++++++++++++++  
\begin{frame}

  \includegraphics[width=0.15\textwidth]{img/ullesc.eps}
  \hspace*{7.5cm}
  \includegraphics[width=0.16\textwidth]{img/fmatesc.eps}
  \titlepage

  \begin{scriptsize}
    \begin{center}
     Facultad de Matemáticas \\
     Universidad de La Laguna
    \end{center}
  \end{scriptsize}

\end{frame}
%++++++++++++++++++++++++++++++++++++++++++++++++++++++++++++++++++++++++++++++  

\begin{frame}
  \frametitle{Índice}  
  \tableofcontents[pausesections]
\end{frame}
%++++++++++++++++++++++++++++++++++++++++++++++++++++++++++++++++++++++++++++++  


\section{Primera Sección}

%++++++++++++++++++++++++++++++++++++++++++++++++++++++++++++++++++++++++++++++  
\begin{frame}

\frametitle{¿Que es \LaTeX{}?}

\begin{definicion}
\LaTeX{} es un sistema de composicion de textos, orientado especialmente a la creación de libros,documentos científicos y técnicos que 
contengan fórmulas matemáticas.

\end{definicion}

\end{frame}
%++++++++++++++++++++++++++++++++++++++++++++++++++++++++++++++++++++++++++++++  

\begin{frame}

\frametitle{Formulas matematicas}

\begin{block}{Ejemplo}
  \begin{itemize}
  \item
  $3x+4x-3x^3=0$ 
  \pause

  \item
  $\int sen(x) dx$
  \pause

  \item
  $\int\int_D x^2$

  \end{itemize}
\end{block}

\end{frame}
%++++++++++++++++++++++++++++++++++++++++++++++++++++++++++++++++++++++++++++++  

\section{segunda seccion}
\begin{frame}
\begin{itemize}
\frametitle{¿Por que utilizar \LaTeX{}?}
Latex es ampliamente utilizado en entornos cientificos. Muchas revistas aceptan documentos escritos en \Latex
Excelente calidad del documento final con salida en distintos formatos:DVI,PDF,...
Los ficheros fuente .tex son ficheros ASCII y pueden ser compilados en cualquier sistema operativo
Es gratuito.
Muy potente
\end{itemize}
\end{frame} 
%++++++++++++++++++++++++++++++++++++++++++++++++++++++++++++++++++++++++++++++  

\begin{frame}
  \frametitle{Bibliografía}

  \begin{thebibliography}{10}

    \beamertemplatebookbibitems
    \bibitem[Plan Estudios, 2011]{plan}  
    Documento de verificación del grado. 
    (2011) 

    \beamertemplatebookbibitems
    \bibitem[Guía Docente, 2013]{guia}  
    Guía docente. 
    (2013) 
    {\small $http://eguia.ull.es/matematicas/query.php?codigo=299341201$}

    \beamertemplatebookbibitems
    \bibitem[URL: CTAN]{latex} 
    CTAN. {\small $http://www.ctan.org/$}

  \end{thebibliography}
  
\end{frame}

\end{document}









